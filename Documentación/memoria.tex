\documentclass[11pt,spanish]{article} % Idioma
\usepackage{babel}
\usepackage[T1]{fontenc}
\usepackage{textcomp, verbatim} % \begin{comment}
\usepackage[utf8]{inputenc} % Permite acentos

\usepackage{wrapfig} % Imagenes %\graphicspath{ {./imagenes/} }
\usepackage[left=2.75cm,top=2.5cm,right=2cm,bottom=2.5cm]{geometry} % Márgenes
\usepackage{amssymb, amsmath, amscd, amsfonts, amsthm, mathrsfs } % Símbolos matemáticos
\usepackage{cancel} % Cancelar expresiones
\usepackage{multirow, multicol, tabularx, booktabs, longtable} % Tablas
\usepackage{fancyhdr, fncychap} % Encabezados
\usepackage{algpseudocode, algorithmicx, algorithm} % Pseudo-código	
\usepackage{bbding} % Símbolos
\usepackage{enumitem} % Enumerados a), b), c)... usando \begin{enumerate}[label=\alph*)]
\usepackage{graphicx, xcolor, color, pstricks} % Gráficos --TikZ-- 
% http://www.texample.net/tikz/examples/
\usepackage[hidelinks]{hyperref}  % Enlaces
\usepackage{verbatim} % Comentarios largos \begin{comment}
\usepackage{rotating} % \begin{rotate}{30}
\usepackage[all]{xy} % Diagramas
\usepackage{xparse} % Entornos
\usepackage{listings}
 
\definecolor{codegreen}{rgb}{0,0.6,0}
\definecolor{codegray}{rgb}{0.5,0.5,0.5}
\definecolor{codepurple}{rgb}{0.58,0,0.82}
\definecolor{backcolour}{rgb}{0.95,0.95,0.92}

\lstdefinestyle{mystyle}{
	backgroundcolor=\color{backcolour},   
	commentstyle=\color{codegreen},
	keywordstyle=\color{magenta},
	numberstyle=\tiny\color{codegray},
	stringstyle=\color{codepurple},
	basicstyle=\footnotesize,
	breakatwhitespace=false,         
	breaklines=true,                 
	captionpos=b,                    
	keepspaces=true,                 
	numbers=left,                    
	numbersep=5pt,                  
	showspaces=false,                
	showstringspaces=false,
	showtabs=false,                  
	tabsize=2
}
\lstset{style=mystyle}
 


% Comandos
\newcommand{\docdate}{}
\newcommand{\subject}{}
\newcommand{\docauthor}{Rubén Morales Pérez}
\newcommand{\docemail}{srmorales@correo.ugr.es}

\newcommand{\N}{\mathbb{N}}
\newcommand{\Q}{\mathbb{Q}}
\newcommand{\C}{\mathbb{C}}
\newcommand{\R}{\mathbb{R}}
\newcommand{\Z}{\mathbb{Z}}


\linespread{1.1}                  % Espacio entre líneas.
\setlength\parindent{0pt}         % Indentación para párrafo.

\title{Proyecto IX Desafío Tecnológico ETSIIT\\
		Dron para monitorización de agricultura y silvicultura }
\author{Ikarus Ants}
\date{\today}

% % % % % % % % % % % % % % % % % % % % % % % % % % % % % % % % %
%					 Inicio del documento
% % % % % % % % % % % % % % % % % % % % % % % % % % % % % % % % %
\begin{document}

\maketitle
\tableofcontents % Generando el indice
\newpage
\setlength\parindent{0pt} % Quitamos la sangría



\section{¿Qué problema intenta resolver su proyecto?}

El objetivo de este proyecto es permitir el análisis de cultivos y/o zonas forestales mediante fotografías con drones aéreos, prescindiendo de la imagen por satélite. Estos permitirán cubrir con mayor detalle (en resolución y cantidad de datos) una zona pequeña o mediana de un entorno natural o de cultivo. 
En el ámbito de cultivo se hará especial énfasis en el cultivo del olivo mediante un análisis de los parámetros del mismo, permitiendo solucionar algunos problemas presentes que perjudican a los productores como la detección a tiempo de enfermedades y plagas. Dicha detección permitiría aumentar el rendimiento de las cosechas y asegurar un estado de salud óptimo en dicha plantación. 

\

En adición, se quiere extender su funcionalidad para solucionar aspectos de importancia en el control forestal como la prevención de incendios, detección de puntos calientes de un incendio, presencia de humo y su posible propagación según la dirección del viento.

\

Se abordarán diversas actividades como la detección de plásticos, mapeo del bosque, estudio de su biodiversidad y su posible aplicación a agricultura de precisión y una silvicultura de precisión que permitiría una explotación más controlada de recursos forestales.


\subsection{¿A qué clientes va destinado?}
Particulares y empresas agrarias de cualquier cultivo, en particular del olivar.
Empresas de gestión de reservas naturales.
Empresas de investigación científica, en especial, en ciencias naturales y biología.


\subsection{¿Qué productos/empresas hacen algo similar?}
\textit{Cegadrone}. Empresa española con sede en Madrid especializado en un gran campo de aplicaciones de drones, siendo dos de sus servicios en el ámbito de agricultura y control forestal. 

\

\textit{Aeromedia}. Empresa española con sede en Málaga, Alicantes y Burgos. Se centra en servicios agro-forestales y en prevención de incendios abordando el problema principalmente con la medición de estados de cortafuegos, informes de la causas de incendios o reconstrucciones de incendios, análisis y valoración de daños.

\

\textit{Recdron}. Empresa española con sede en Murcia y Granada centrada en producción audiovisual con un servicio secundario de agricultura de precisión. Ofrece vuelos periódicos a fincas para dar recomendaciones para mejorar los cultivos según los datos recogidos.  

\

\textit{Zenit drones}. Empresa de drones de Motril centrado en agricultura de precisión, inspección industrial, termografía y audiovisual.

\

\textit{Tvant}. Empresa de drones de Córdoba que da servicio a toda Andalucía. Se centran en el sector de la agricultura de precisión. Precio variable según la tarea a solucionar.

\subsection{¿En qué se diferencian esos productos/empresas de su propuesta?}
Se ofrece un servicio periódico asequible sin requerir una gran inversión inicial. 
Instalación de una estación en el terreno, cuyo papel es la recarga de los drones y transmisión de datos a la nube.
Vuelo autónomo, puede funcionar de forma autónoma sin necesidad de personal especializado.
Interfaz fácil y accesible desde una aplicación web.
Recolección y tratamiento de datos, que informan al cliente del estado de su terreno y cultivos.

\subsection{¿Cuáles son los costes más importantes por fabricación de cada unidad, mantenimiento e instalación?}

Dron con capacidad de vuelo autónomo.
Sistema de cámara multiespectral y térmica y unidad de procesamiento de datos.
Base del dron para recarga automática, comunicación con servidores y recarga solar.
Coste adicional: infraestructura de red para la comunicación y monitorización.

\subsection{¿Cómo obtendrá beneficios del producto (por adquisición de cada unidad, por mantenimiento, por instalación, por análisis de datos que genera...)?}
Proveemos un servicio, nuestra empresa se reserva el derecho de propiedad del sistema autónomo aunque se encuentre en la instalación de cultivo del cliente. La empresa realiza la instalación y mantenimiento activo el servicio mediante un coste mensual, trimestral o anuales asequible.

\subsection{Business Model Canvas}


\section{Diseño inicial de la propuesta}

\subsection{¿En qué entorno se puede aplicar su producto?}
Aplicación directa en espacios naturales abiertos como campos de cultivo, bosques o grandes plantaciones. También resulta útil en reservas forestales, espacios protegidos y en general cualquier sitio con vegetación que se quiera monitorizar.


\subsection{Describa un ejemplo de caso de uso de su producto (cómo funcionaría desde el punto de vista del usuario)}

Acceso a una interfaz intuitiva que permite gestionar uno o más drones instalados con sus respectivas bases. La interfaz proporciona una imagen aérea de los cultivos y permite aplicar diferentes filtros en la interfaz para visualizar diferentes parámetros de cultivo. Existe una página principal que proporciona datos generales de dicho cultivo. A partir del análisis de dichos datos se generarán una serie de advertencias y/o sugerencias y una planificación de misiones para los drones. Dichas misiones pueden realizarse de forma autónoma cada cierto tiempo o activarse manualmente, según la preferencia del cliente.

\subsection{Enlace a un diagrama de su sistema (¿qué elementos básicos lo compondrán?)} 
% TODO: rellenar

\subsection{Describa brevemente qué elementos componen sus sistema, y qué función tienen}

\begin{figure}
	\centering
	\includegraphics[totalheight=8cm]{img/esquema_inicial.png}
	\caption{Esquema el sistema}
	\label{fig:esquema_inicial}
\end{figure}

Los 6 componentes de nuestro sistema (Figura \ref{fig:esquema_inicial}) son:
\begin{itemize}
\item \textbf{Dron Aéreo}: se encarga de volar de manera autónoma por la zona a monitorizar, llevando consigo la raspberry pi y los sensores.

\item \textbf{Raspberry Pi}: pequeño ordenador encargado de recoger los datos con sensores que lleva conectados.

\item \textbf{Sensores}: cámara multiespectral y térmica. 

\item \textbf{Estación}: base cubierta que sirve de resguardo al dron. Además, es el punto de envío de los datos recogidos a los servidores y carga de la batería del sistema de recolección de datos (dron, raspberry pi  y sensores)

\item \textbf{Servidores}: procesar y almacenar la información que ha sido enviada desde el lugar de monitorización.

\item \textbf{Interfaz web}: disponible para que el usuario pueda consultar los parámetros de interés sobre la monitorización de su terreno.
\end{itemize}

\subsection{Indique qué tecnología piensan usar para implementar los elementos del sistema.}
El software a utilizar será el código necesario para el control del cuadricóptero durante su vuelo usando una Raspberry Pi B+. La sección de hardware consistirá en el ensamblaje de los componentes individuales necesarios para el vuelo y la navegación así como usar Navio+ para pruebas. Los componentes no los seleccionaremos en un kit sino que lo haremos individualmente. La parte de toma de imágenes será llevada a cabo con una Raspberry Pi 3 y librerías de procesamiento de imágenes. Para el funcionamiento en conjunto del sistema usaremos RTI Connext DDS y será necesario realizar una interfaz web para el usuario.


\section{Características}
Los datos a medir en nuestro caso pueden ser [5]:

\begin{itemize}
	\item Imágenes geo-etiquetadas visibles y multiespectrales de la zona reconocida.
	\begin{itemize}
		\item Altura de las plantas
		\item Número de plantas
		\item Salud y estado de las plantas
		\item Presencia de nutrientes
		\item Presencia de enfermedades
		\item Presencia de malas hierbas
		\item Estimación de biomasa
		\item Datos volumétricos
	\end{itemize}

	\item Rendimiento de maquinaria agrícola como cosechadoras, sembradoras, tractores, etc.
\end{itemize}


Recapitulando y conceptualmente, los ámbitos en el que se aplica nuestra solución son los siguientes:	
\begin{itemize}
	\item Monitorizado de cultivos:
	\begin{itemize}
		\item Inspección o búsqueda de patologías desde el aire
	\end{itemize}
	\item Topografía del terreno a partir de la fotogrametría.
	Para un mayor control sobre irregularidades del terreno es necesario conocer su topografía.
	\item Control forestal
	\begin{itemize}
		\item Detección de basura y posible recogida.
		Una prevención total del incendio, implica prevenir una de los causas más importantes de los incendios forestales
		\item Prevención de incendios y ayuda especializada en su extinción
		\item Control o monitorizado de animales en peligro de extinción
		\item Búsqueda de personas desaparecidas en caso de emergencia
	\end{itemize}
	
\end{itemize}


\subsection{Necesidad de esta solución (¿qué no aportan otras?)}
% TODO
Los trabajadores de cultivos han de saber el estado del terreno:
\begin{itemize}
	\item Los terrenos pueden ser enormes, es costoso recorrerlos e ir examinando 
	\item Podríamos ofrecer una visión global del terreno cada x tiempo
	\item Reconocimiento de zonas en las que es necesario una acción (cuidados especiales para una zona del terreno o recoger basura)
\end{itemize}

Sensores en el terreno:
\begin{itemize}
	\item Posible avería ante condiciones adversas de parte de los sensores que tienen que estar 24/7 ahí % TODO refrasear
	\item Hay que alimentar los sensores (si no tienen placas fotovoltaicas habría que cablearlo y si las tienen sería caro poner muchos)
	\item No advierten de la basura en el terreno
\end{itemize}


\subsection{Descripción del producto}
Requerimientos Hardware % TODO: Jesús, José Miguel

\begin{itemize}
	\item Requerimientos aéreos del dron
	\item Funcionamiento de las cámaras para medir la salud de una plantación
\end{itemize}

La mayor parte de la agricultura depende de imágenes multiespectrales. Específicamente monitorizando los cambios en el tiempo de la luz VIS (Visible Light) y NIR (Near Infrared Light) reflejada por los cultivos ya que las plantas reflejan diferentes cantidades de estos dos tipos de luz dependiendo de su salud. Véase la siguiente imagen  % TODO [7]


\begin{figure}{H}
	\centering
	\includegraphics[totalheight=8cm]{img/leaf.png}
	\caption{Reflejo de luz}
	\label{fig:luz}
\end{figure}

Debido a las capas esponjosas de las hojas en su parte posterior, estas reflejan una gran cantidad de luz en el espectro cercano infrarrojo en contraste con otros objetos diferentes a las plantas. Cuando una planta está deshidratada o estresada, esa capa esponjosa colapsa y reflejan una menor cantidad de infrarrojo aunque la misma en el espectro visible. Comparando estas dos señales podemos diferenciar una planta sana de una enferma en lo que se llama NDVI (Normalized Difference Vegetation Index) [8] y se calcula como

\[ NDVI=(NIR-VIS)/(NIR+VIS) \]

Otros parámetros a considerar pueden ser

\begin{itemize}
	\item CWSI (Crop Water Stress Index):
	Mide las diferencias de temperatura para detectar o predecir estrés de agua en la planta. Requiere de un sensor térmico y el uso de una estación meteorológica cercana.
	\item CCCI (Canopy Chlorophyll Content Index): detecta los niveles de nitrógeno de usando una tri-banda cerca del espectro visible del rojo.
\end{itemize}


Centrándonos en la cámara se requiere de una cámara de alta resolución que tome imágenes del espectro visible (VIS). La resolución mínima para agricultura es de unos 12 megapíxeles. Otra opción es usar cámaras con un mayor ángulo de visión u ojo de pez que permite una captura mayor por imagen. Con un procesado posterior de imagen podemos eliminar la distorsión causada por este tipo de cámaras.

\

Por otro lado, las cámaras multiespectrales nos permiten ver bandas del espectro diferentes a la visible. 
Cámara multiespectral para DJI Phantom https://sentera.com/dji-ndvi-upgrade/

\


Otro tipo de cámaras son las térmicas que permiten ver puntos calientes y medir los cambios de temperatura de las plantas con el paso del tiempo. Permiten también detectar la presencia de agua, útil en caso de inundaciones.

\

LIDAR es otra forma de medición de precisión que mide la distancia iluminando el objetivo con un láser y analizando la luz reflejada. Sirve para la medición y realización de modelos en 3D precisos. (Son los sensores más caros +60.000\$)

\

Requerimientos de Software% TODO: Carlos, Alex
\begin{itemize}
	\item Frontend: Web o app
	\item Backend: Software del dron + Software de la rpi (si la usamos para los sensores) + Software de la estación (se encarga de recoger los datos y adecuarlos al frontend)	
\end{itemize}

\subsection{Modelo de negocio}
% TODO: hacer

%%%%%%%%%%%%%%%%%%%%%%%%%%%%%%%%%%%%%%%%%%%%%%%%%%%%%%%%%%%%%%%%%%%%%%%%%%%%%%%%%%

% % % % % % % % % % % % % % % % % % % % % % % % % % % % % % % % %
%					 Bibliografía
% % % % % % % % % % % % % % % % % % % % % % % % % % % % % % % % %

\newpage
\bibliography{citas} %archivo citas.bib que contiene las entradas 
\bibliographystyle{plain} % hay varias formas de citar

\end{document}
